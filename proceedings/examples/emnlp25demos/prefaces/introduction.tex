Welcome to the Sixth Workshop on Privacy in Natural Language Processing. Co-located with NAACL 2025 in Albuquerque (NM), USA, the workshop is scheduled for April 4, 2025. To facilitate the participation of the global NLP community, we continue running the workshop in a hybrid format.
\bigskip


Privacy-preserving language data processing has become essential in the age of Large Language Models (LLMs) where access to vast amounts of data can provide gains over tuned algorithms. A large proportion of user-contributed data comes from natural language e.g., text transcriptions from voice assistants. It is therefore important to curate NLP datasets while preserving the privacy of the users whose data is collected, and train ML models that only retain non-identifying user data.
The workshop brings together practitioners and researchers from academia and industry to discuss the challenges and approaches to designing, building, verifying, and testing privacy preserving systems in the context of Natural Language Processing.
\bigskip


Our agenda features a keynote speech, hybrid talk sessions both for long and short papers, and a poster session. This year we received 13 submissions. We accepted 9 submissions after a thorough peer-review. One accepted submissions has been withdrawn by the authors.
\bigskip


We would like to deeply thank to all the authors, committee members, keynote speaker, and participants to help us make this research community grow both in quantity and quality.
\bigskip

Workshop Chairs